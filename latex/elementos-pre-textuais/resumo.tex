Os sistemas de automação industrial controlam seus processos utilizando de controladores lógicos programáveis (CLPs), na grande maioria das aplicações, e costumeiramente conjugados com interfaces homem-máquina (IHMs), para acesso às suas informações e do processo controlado.
O laboratório de controle de processos, do IFSP campus Salto, possui CLPs e IHMs profissionais montados em kits didáticos, possibilitando o estudo teórico e prático de suas aplicações industriais, porém, 
a ausência de manuais de utilização e de conexão entre os equipamentos, dificulta a sua utilização de forma plena e habitual nas disciplinas pertinentes nos cursos de Engenharia de Controle e Automação e Técnico em Automação Industrial.
Deste forma, este trabalho assume o objetivo de produzir um manual de comunicação entre equipamentos e seus primeiros passos de operação, e 
utilizando a interface RS-485 e o protocolo  de comunicação Modbus, por ser um dos mais bem estabelecidos meios de comunicação entre equipamentos industriais e não uma interface específica deste cenário, garante-se assim a sua abrangência. 


% Separe as palavras-chave por ponto e vírgula ';' e finalizadas por ponto.
\palavraschave{Comunicação industrial; RS-485; Modbus; IHM; CLP.}
