\chapter{Fontes Principais de Informação}
\label{cap:trabalhos-relacionados}

Como fontes primárias de informações sobre o Terminal gráfico iX-T7F-2 (\acrshort{IHM}) e \acrshort{TB}131 (CLP DUO), 
recomenda-se a consulta do seu material de apoio, 
costumeiramente muito bem documentado, 
através do Site Oficial (www.altus.com.br) ou ainda das redes sociais como o Linkedin e Youtube,
em que são disponibilizados artigos relacionados aos seus equipamentos e tecnologias envolvidas. 


\begin{figure}[ht!]
	\centering
	\Caption{\label{fig:altus-linkedin} Canal da Altus S.A. no Linkedin}
	\UECEfig{}{
		\fbox{\includegraphics[width=12cm]{figuras/altus-linkedin}}
	}{
		\Fonte{Linkedin Altus \cite{altus-linkedin}}
	}
\end{figure}


No canal do Linkedin da Altus podem ser encontradas informações sobre a empresa, artigos relacionados à industria e seus ramos de atuação, assim como sobre equipamentos e suas aplicações. Ainda é possivel visualizar oportunidades de emprego oferecidas pela empresa.


\begin{figure}[ht!]
	\centering
	\Caption{\label{fig:altus-youtube}Canal da Altus S.A. no youtube}
	\UECEfig{}{
		\fbox{\includegraphics[width=12cm]{figuras/altus-youtube}}
	}{
		\Fonte{Youtube Altus \cite{altus-youtube}}
	}
\end{figure}


No canal da Altus no Youtube estão concentrados os tutoriais para as diversas linhas de equipamentos, através de vídeos curtos e linugagem objetiva, os tutoriais oferecem uma forma prática de desenvolver o aprendizado na configuração, programação dos equipamentos da empresa. Também podem ser encontrados Webinars sobre os mais variados assuntos de tecnologia, principalmente envolvendo as técnoclogias mais atuais como Internet das Coisas - IoT (\textit{Internet of Thongs}), Segurança nas redes industriais, Sistemas de Controle e Supervisão, entro outros assuntos que fazem parte do universo Altus. 


A Beijer Electronics 
como empresa desenvolvedora de software e hardware das \acrshort{IHM}s, disponibiliza tutoriais para a sua utilização, com a vasta gama de recursos oferecidos pelo equipamento.


\begin{figure}[ht!]
	\centering
	\Caption{\label{fig:beijer-youtube}Canal da Beijer Electronics no Youtube}
	\UECEfig{}{
		\fbox{\includegraphics[width=8cm]{figuras/beijer-youtube}}
	}{
		\Fonte{Youtube Beijer Electronics \cite{beijer-youtube}}
	}
\end{figure}





Além dos canais oficiais, muitos outros podem ser acessados contendo informações, tutoriais, artigos sobre as tecnologias e sobre o uso dos equipamentos, 
com as mais variadas didáticas, que talvez te agrade mais. 
Tomando as devidas precauções quanto a legitimidade das informações e da seriedade do autor, explore as possibilidades, mas sem deixar de ter como referência as fontes originais. 




